\documentclass[11pt,a4paper]{article}
\usepackage{enumitem}

\SetEnumitemKey{seclist}{label=\thesection.\arabic*,left=0pt}
\SetEnumitemKey{subseclist}{label=\thesubsection.\arabic*,left=0pt}
\SetEnumitemKey{subsubseclist}{label=\thesubsubsection.\arabic*,left=0pt}
\SetEnumitemKey{subsecsublist}{label*=.\arabic*,left=-\leftmargin}
\SetEnumitemKey{subsubsecsublist}{label*=.\arabic*,left=-\leftmargin}

\begin{document}

\title{CASA VLBI Requirements} \author{Mark Kettenis
  \footnote{Based on an earlier document written by Walter Brisken,
    Vincent Fish, Jeff Kern and Shep Doeleman}}
\maketitle

\section{Introduction}

Over the last three years, a software development team from JIVE, in
collaboration with the CASA team at NRAO has worked on adding VLBI
capabilities to CASA.  Basic VLBI data reduction is now possible in
CASA and this functionality has been successfully used in the
rPICARD\footnote{M. Janssen et al, rPICARD: A CASA-based calibration
  pipeline for VLBI data, A\&A 626, A75 (2019)} pipeline that was used
for verification of the recent Event Horizon Telescope results.  But
while a \texttt{fringefit} task is now available in CASA, some of the
bells and whistles available in older packages like AIPS are still
missing.  In addition to that there still is other functionality that
is used for more advanced VLBI data reduction that is not yet
available in CASA.  As such CASA is not yet a fully functional
replacement for AIPS.

This document is an attempt to identify what functionality is still
missing.  It collects stakeholder input collected at various training
events and discussions with the rPICARD devloper with requirements
from an earlier NRAO document written by Walter Brisken, Vincent Fish,
Jeff Kern and Shep Doeleman.  In fact a large fraction of the
requirements in this document and some of the motivations for those
requirements were taken verbatim from that document.  Some of the
requirements have been dropped even if the functionality has not been
implemented in CASA because it was felt that those requirements were
better addressed in a pipeline that uses CASA as a basis.  In fact a
lot of the functionality related to incoherent fringe fitting has been
implemented already in the rPICARD pipeline.  Requirements specific to
geodesy have also been dropped for now as we consider them out of
scope for the current devlopment effort at JIVE.  These requirements
may be brought back at a later stage.

Our intention is to share this document with stakeholders and use
their input to prioritize future development efforts.  We intend to
update this document with that input.


\section{Generic}

This section identifies some requirements that are related to the
general calibration infrastructure in CASA.

\begin{enumerate}[seclist]

\item Interpolation during application of various calibration tables
  (including, but not limited to fringe fitting) should have the
  capability of respecting scan boundaries.

\item Completed for CASA 6.1.

\item Geometry-dependent calibration (parallactic angle, gaincurve,
  opacity, etc.) should use the POINTING.DIRECTION instead of
  FIELD.DELAY{\_}DIR / PHASE{\_}DIR / REFERENCE{\_}DIR.

  Question: How does mean vs. apparent position factor into this?

  Answer: These two coordinates only differ substantially when
  pointing near the horizon.  For dish antennas the gain curves tend
  to be very flat at low elevations so this isn't important in that
  case.  However for aperture arrays the effect may be considerable.

\item It shall be possibe to add intents to an existing Measurement Set.
  
\end{enumerate}


\section{Amplitude calibration}

Amplitude (gain) calibration for VLBI is done quite differently from
other types of arrays since typically there are no usable flux
calibrators at VLBI resolution.  In particular the traditional gain
calibration method using the \texttt{setjy} task cannot be used for
VLBI.


\subsection{System temperatures}

\begin{enumerate}[subseclist]

  \item It shall be possible to use gain constrained self calibration
    to solve (at least approximately, on bright sources) the
    appropriate gains on a specified set of antennas to work around
    cases where system temperature data is either not available or is
    not reliable for all antennas. The ANTUSE parameter in AIPS tasks
    CALIB is used for this purpose.

    Note: selfcal is available in CASA but there is no equivalent of
    the ANTUSE parameter.

\end{enumerate}


\subsection{Opacity correction}

\begin{enumerate}[subseclist]

\item It shall be possible to specify a zenith opacity.

  Note: This may already be possible in \texttt{gencal}.

\item It shall be possible to use system temperature measurements made
  over a specified period as input to solve for zenith opacity.

  \begin{enumerate}[subsecsublist]

  \item For antennas with characterized spill-over and metrology data,
    the ground temperature should be estimated and removed before
    solving for zenith opacity.

  \item It shall be possible to specify an elevation range for data to
    fit.

  \item Both least squares and a robust fitting algorithm shall be available.

  \item Given zenith opacity, possibly as a function of time,
    correction for opacity based on sec(z) shall be possible.
    
  \end{enumerate}
  
\end{enumerate}

Note: This functionality is present in rPICARD.  Integrating that code
into CASA is probably desirable.


\subsection{Antenna gain}

\begin{enumerate}[subseclist]

\item Completed for CASA 6.2

\item Completed for CASA 6.2

\item It should be possible to plot gain curves as function of elevation.

\item It should be possible to plot gain curves as function of time
  (following antenna pointing).

  Note: There is a wish to support such a feature generally, plotting
  the calibration as a function of the relevant interpolation axes in
  the MS (time, freq).  The option of plotting the calibration
  parameters or their effect (the elements of the Jones matrix) may be
  desirable.

\end{enumerate}

Gain curves are now automatically imported when using the
\texttt{importfitsidi} task when GAIN{\_}CURVE tables are present in
FITS-IDI files.  Helper scripts are available to create gain curve
tables from text files in ANTAB/VLOG format.  These scripts may not
always work correctly for dual-band observing.


\subsection{Decorrelation correction}

\begin{enumerate}[subseclist]

\item Correlator averaging parameters, possibly as a function of
  baseline, should be stored with the visibility database.

\item It shall be possible to optionally correct for decorrelation
  without requiring an additional pass through the data (on-the-fly
  application).

\item This correction shall be based on the cumulative delay/rate
  calibrations being applied.  This correlation should only be applied
  once as it can't be done properly in a cumulative fashion.

\end{enumerate}


\subsection{Calibration editing}

\begin{enumerate}[subseclist]

\item It shall be possible to manually edit input calibration data.

  Note: The \texttt{plotms} and \texttt{plotcal} tasks already support
  manual flagging of calibration tables.

\end{enumerate}


\section{Pulse Cal data handling}

Phase (or equivalently, pulse) calibration is a VLBI technique used to
correct the sampled data for instrumental effects. For example, in
geodesy one would like the observations to measure the baseline delay
to a fixed point on each antenna, typically the intersection of
axes. However, the data are sampled on the ground after passing
through cables, connectors, down-converters, and filters.  By injecting
a series of pulses as close to the front end as is practical, a series
of tones is produced in the frequency domain, which can be used to
solve for both delay and phase effects between the front end and the
samplers.

For example, in the broadband system used for the NASA Space Geodesy
Program, pulses are injected at a 5MHz rate, producing tones (or
rails) at 5MHz intervals.  These tones are extracted in the
correlation software and written to a calibration file that
accompanies the visibility data, one per antenna. The phase cal data
consist of triplets of frequency, amplitude, and phase, tabulated
every second.

The fringe-fitting software (e.g., HOPS FOURFIT) finds via FFT the
best-fit line of phase as a function of frequency, using all of the
tones in a channel or any desired subset thereof (in the case where
known RFI corrupts tones). The slope determines a delay, which is then
differenced on the baseline and applied to the complex visibility
data.  The visibilities are also adjusted by the differential phase
(at mid-band) of the two fits.

This process allows data that have passed through different
anti-aliasing filters and samplers to be registered with one another,
thus allowing phase-coherent delay solutions across multiple wide IFs.
This technique has been applied with success to group delay extraction
over a frequency span of about 6 GHz.

The comb frequency structure causes ambiguities in measured delay; any
delay measurement solely determined by a comb with frequency interval
Q can only determine delay modulo 1/Q. Resolution of this ambiguity
can come from fringe fitting some data. Usually only a very small
amount of data for an entire experiment is required as delays
typically don’t change by more than 10s of nanoseconds and the
ambiguities are typically 200 or 1000 nanoseconds. Continuity of delay
through time can be used to extend the period of ambiguity resolution.

\subsection{Data Structure}

\begin{enumerate}[subseclist]

\item Import of Pulse Cal data shall be supported

\begin{enumerate}[subsecsublist]

\item Pulse cal data attached to a FITS file shall be importable.

\item Each pulse cal measurement contains a real and imaginary value
  and the time interval corresponding to that measurement.

\item A pulse cal set is the collection of all pulse cal measurements
  made over one time interval at one antenna.

\end{enumerate}

\item Between 0 and B+1 pulse cal tones per sub-band must be supported
  where B is sub-band bandwidth in MHz.

\item A cadence as fast as one pulse cal set per visibility
  integration time should be supported.

  Note: This requirement contradicts the example system given above
  and probably needs to be changed.  Maybe ``faster than one pulse cal
  set per visibility''?

\item Time averaged pulse cal data should be supported; averaging
  intervals may be integer multiples of the visibility integration
  time or not.

\item An optional ``cable cal'' value, containing an additional
  instrumental delay correction, should be handled along with pulse
  cal data.

\item Different antennas may have different pulse cal intervals and/or
  number of tones.

\item Single precision floating point is sufficient for the real and
  imaginary parts of each pulse cal measurement; time should be
  accurately representable to at least 1ms.

\end{enumerate}

\subsection{Pulse Cal Data Selection}

\begin{enumerate}[subseclist]

\item It should be possible to plot a time series of pulse cal
  amplitude or phase as a function of time for a selection of
  tones. Similarly the cablecal values should be plottable.

\item It should be possible to view the amplitude or phase of the time
  series of a pulse cal set as a raster image.

\item It should be possible to flag certain pulse cal values based on
  a priori information or interactive editing processing; flagged
  values should be ignored in computations involving the pulse cal
  data.

\end{enumerate}

\subsection{Calculations to perform}

\begin{enumerate}[subseclist]

\item It shall be possible to determine the delay as a function of
  time based on the Pulse Cal data.

\begin{enumerate}[subsecsublist]

\item Solutions should be determined separately for each antenna and
  separately for each spectral window.

\item It shall be possible to determine a single delay value from
  multiple subbands.

\end{enumerate}

\item In cases where cable cal data is present there should be the
  option to include the cable cal correction in the computed delay.

\item The determined delays should be stored in a table that can be
  further edited and applied as necessary.

\item It shall be possible to time average pulse cal values.

\item It shall be possible to form a bandpass calibration table based
     on pulse cal sets.

\item It shall be possible to form a gain calibration table by
  extracting the amplitude and/or phase of a single tone of each
  spectral window.

\item It shall be possible to use fringe-fit determined delays to
  resolve pulse cal delay ambiguities.

Note: This has been partly prototyped in Python by Des Small.

\end{enumerate}


\section{Fringe fitting}

\subsection{General Delay Fitting Requirements}

\begin{enumerate}[subseclist]

\item It shall be possible to determine and correct cross-polarized
  delays and phases.

\begin{enumerate}[subsecsublist]

\item Completed for CASA 6.1

\item Completed for CASA 6.1

\end{enumerate}

\item Completed for CASA 6.2

\end{enumerate}


\subsection{Modes of operation}

Additonal modes of operation shall be implemented for the
\texttt{fringefit} task.

\begin{enumerate}[subseclist]

\item It shall be possible to overlap time ranges by a specified
  amount (see the SOLSUB parameter for AIPS task \texttt{FRING}).

\item In cases where the cadence does not evenly span the data valid
  period of a scan an intelligent algorithm to shift the intervals
  shall be invoked.

\item It shall be possible for delay and rate windows, both center and
  width, to be specified by the user on a per-antenna basis.

\item It should be possible to smooth calibration solutions obtained
  using the CASA \texttt{fringefit} task.

\begin{enumerate}[subsecsublist]

\item It should be possible to smooth parameters of these solutions
  (phase, delay, rate) individually.

\end{enumerate}

\item It should be possible to combine data by correlation products
  prior to determination of the delay paramters.

  Question: Should this combine just the parallel hands?  Stacking
  cross-hands could be helpful when working on data with poor
  polarization leakage.  Should this be a simple stacking or is this
  better done as a simultaneous fit?  The goal here is additional S/N
  and Walter Brisken suggests averaging visibilities across
  polarisations.

\item It shall be possible to use alternative weighting schemes in
  the CASA \texttt{fringefit} task (see the \texttt{WEIGHTIT}
  parameter for AIPS task \texttt{FRING}).  At least the standard
  weighting by $1/\sigma^2$, weighting by $1/\sigma$ and no weighting
  (weighting by $1$) shall be supported.

\item It shall be possible to use baseline stacking techniques to
  assist fringe detection.

\item It shall be possible to combine spectral windows (in order to do
  a multi-band fit) even if there are large gaps (in frequency)
  between spectral windows.

\item It shall be possible to combine spectral windows even if the
  channels of different spectral windows do not align on a single
  frequency grid.

  Note: A possible approach has been prototyped in Python by Des Small.

\item An implementation of Fringe-rate mapping shall be provided.

  Note: This is required for maser sources where it is hard/impossible
  to get a good starting postion.  The algorithm implemented in AIPS
  is sufficient but cumbersome to use.  More advanced algorithms
  exists but may be out of scope.

\end{enumerate}


\section{Polarization}

The following requirements relate to the handling of data correlated
with full polarization (i.e., all 4 polarization products).

\subsection{ Polatization calibration}

\begin{enumerate}[subseclist]

\item It must be possible to determine and correct for antenna
  polarization leakages including using \emph{resolved} polarization
  calibrators.

  Note: Prototype implementation exists (by Ivan Marti-Vidal) in
  (mostly) Python.

\item It must be possible to apply the (co)-parallactic angle
  correction for X-Y mounts.  This is necessary to support calibration
  of data that includes the Hobart 26m antenna which is part of the
  Australian Long Baseline Array (LBA).

  Note: Implementation (for casacore) exists.

\end{enumerate}


\subsection{Calibration}

\begin{enumerate}[subseclist]

\item It shall be possible to transfer delay calibration information
  across polarizations to increase coherence time:

\begin{enumerate}[subsecsublist]

\item It shall be possible to use fringe solutions in one polarization
  (e.g., RR) to assist in fringe detection in another polarization
  (e.g., RL) via baseline stacking techniques.

\item It shall be possible to use the visibility phases as a function
  of time on a baseline in one polarization to attempt to extend the
  coherence time in another polarization.

\end{enumerate}

\end{enumerate}


\section{Model accountability and manipulation}

Note: The original introduction here scetched an ambition to implement
geodetic capabilities into CASA to allow absolute astrometry.  This is
not the ambition of the current work that is being done.  However some
of these aspects are relevant for good relative astrometry support as
well.

\subsection{Delay model propagation}

\begin{enumerate}[subseclist]

  \item Throughout processing the delay model shall be stored in
    conjunction with the data, further the delay model shall be kept
    consistent with the current state of the data.

  \item Delay model versioning must be maintained even after splitting
    and recombining measurement sets.

  \item Throughout processing delays shall be represented to at least1
    femtosecond resolution.

  \item It shall be possible to maintain separation of delay effects
    (i.e., vacuum propagation, atmospheric terms, etc.) within the
    delay model.

  \item It must be possible to store the delay model as a polynomial
    spline. Polynomials with up to 6 terms must be supported. The
    interval of validity of a specific polynomial must be stored and
    can range from 1 second to 1 hour. A given polynomial must be
    identified with a particular source.

  \item The delay model used during correlation shall be imported with
    the visibility data..

  \item All calibrations containing a delay shall modify the delay
    model table to ensure consistence with the data, in particular the
    total delay shall be unchanged.

  \item During any operation where a new visibility database is
    formed, a delay model table consistent with the state of the new
    database shall be written

\end{enumerate}

Note: Some of these requirements are not compatible with the way CASA
operates and/or fundamental design principles of the MeasurementSet.
It is not unlikely that these requirements will need adjustment.
These requirements are maintained for now as some of them will be
needed to implement calibrations mentioned in the next section.

\subsection{Delay model adjustments}

Earth Orientation Parameters (EOPs) are used to describe the
orientation and spin phase of the earth relative to a standard model
of a uniformly spinning orb. The deviations from uniform motion are
unpredictable as they are largely driven by transfer of angular
momentum between the earth's crust, the oceans, atmosphere and earth
core. Typically final best estimates for the EOPs are only available a
week or two after observation, which may be after correlation.

\begin{enumerate}[subseclist]

\item It shall be possible to apply delay corrections to the data with
  improved EOPs\footnote{The equivalent functionality is implemented
    in AIPS task \texttt{CLCOR} when using \texttt{OPCODE = 'EOPS'}.}.

  Note: Current thinking is that it would be better to recalculate the
  model using updated EOPs instead of attempting to do a differential
  correction based on the difference in EOPs

\item A mechanism for automatically updating EOP data should be
  implemented.

  Note: VLBI correlators tend to use a different EOP data product than
  the IERS EOPs that are already present in the CASA data repository.

\item Source position adjustment should be possible.

  \begin{enumerate}[subsecsublist]

  \item New source positions shall be specified in J2000 frame coordinates.

  \item The applied correction shall produce phases equivalent to those that would come from.
    correlation with the new source coordinates.

  \item The coordinates of the correlation center in the data set shall be updated.

  \item The baseline vectors (U,V,Ws) shall be updated to be consistent with the new correlation
    center.

  \item Source position corrections up to 1 arcminute should be supported.

  \end{enumerate}

  Note: While this functionality is in principle implemented in the
  \texttt{fixvis} task, there is a strong suspicion that the
  underlying model calculations may not be accurate enough.
  
\end{enumerate}


\subsection{Delay model replacement}

\begin{enumerate}[subseclist]

\item It shall be possible to derive a delay correction from the
  difference between the current delay model table and an external
  delay model.

\end{enumerate}


\section{Miscellaneous requirements}

\subsection{Ionospheric Correction}

\begin{enumerate}[subseclist]

\item It shall be possible to derive a dispersive delay calibration
  table from external ionosphere data. (Equivalent to AIPS task
  \texttt{TECOR})

\item External ionosphere data in IONEX format should be downloaded on
  demand from cddis.gsfc.nasa.gov. (For reference implementation, see
  AIPS task \texttt{TECOR})

\end{enumerate}


\appendix

\section{Completed Requirements}

\subsection{Generic}

\begin{enumerate}[subseclist]

\item Visibility data should be flagged in a polarization-independent
  fashion if desired by the user (currently CASA flags all
  polarization products if one of them ends up being explicitly or
  implicitly flagged).  (was 2.2, completed for CASA 6.1)

\end{enumerate}


\subsection{Amplitude calibration}

\subsubsection{Autocorrelation corrections}

\begin{enumerate}[subsubseclist]

\item A facility to divide cross-correlation values by the geometric
  mean of the associated and time- coincident autocorrelations is -
  required.

  Note: implemented by accor.

\item A mechanism to plot the time variability of sub-band average
  autocorrelations would be useful.

  Note plotms can do this already.
  
\end{enumerate}

\subsubsection{System temperature}

\begin{enumerate}[subsubseclist]

  \item System temperature tables imported with visibility data shall
    be able to be applied to the data and weights.

    Note: implemented in gaincal.

  \item It should be possible to import system temperature data in the
    standard ``TSM'' format (See AIPS task ANTAB)

    Note: CASA expects this data to be present in the
    SYSTEM\_TEMPERATURE table of the FITS-IDI file upon import.  Tools
    exist to append this data to the FITS-IDI file.  These tools
    accept both ``standard'' AIPS ANTAB format and VLBA VLOG format.

\end{enumerate}


\subsubsection{Antenna gain}

\begin{enumerate}[subsubseclist]

\item Elevation gain curve tables naturally imported with visibility
  data should be usable.  (was 3.3.1, completed for CASA 6.2)

\item Application of different elevation gain curves on different
  sub-bands should be supported.  This is especially important in
  cases where dual-band observing (e.g., S- band and X-band
  simultaneously) is employed.  (was 3.3.2, completed for CASA 6.2)

  Note: This can already be done in CASA because gain curves are
  realized and resolved per spectral window.  There may be ambiguities
  that need to be resolved though.

\end{enumerate}


\subsubsection{Digital corrections}

\begin{enumerate}[subsubseclist]

\item It shall be possible to apply the Van Vleck (and its >1-bit
  equivalents) to the visibility data.

  Note: implemented in importfitsidi

\end{enumerate}


\subsection{Fringe Fitting}

\subsubsection{General Delay Fitting Requirements}

\begin{enumerate}[subsubseclist]

\item It shall be possible to determine singleband delay, multiband
  delay, and rate solutions on a single baseline.

  Note: While this requirement is met, more work may be necessary to
  make its use practical as HOPS-style baseline-based fringe fitting
  isn't fully implemented.

\item It shall be possible to determine global antenna-based
  single-band delay, multiband delay, and rate solutions on an array
  of baselines or subset thereof.

  Note: implemented in fringefit.

\item It shall be possible to include in the fit a dispersive
  multi-band delay component proportional to $1/\nu^2$. This is
  relevant for ionospheric calibration.  (was 5.1.1.1, completed for CASA 6.1)

\item It shall be possible to solve for any combination of delay,
  dispersive delay, rate and phase, where unsolved parameters are
  assumed to be zero.  This should include the possibility to solve
  for rate if only a single frequency channel is present.  (was
  5.1.1.2, completed for CASA 6.1)

  Note: This is different than zeroing solutions after solving.
  
  Note: Des is of the opinion that excluding the phase makes no sense.

\item It shall be possible to plot fringefit solutions using the
  \texttt{plotms} task.  (was 5.1.2, completed for CASA 6.2)

\end{enumerate}

\subsubsection{Selection}

Note: Full power of CASA's data selection interface is availabe in fringeit.

\subsection{Modes of operation}

\begin{enumerate}[subsubseclist]

\item It must be possible to specify a source model for fringe fitting.

  Note: Generic CASA mechanism to divide out a model is available in
  fringefit.

\item It must be possible to specify a desired solution cadence
  (equivalent to the SOLINT parameter in the AIPS task FRING).

  Note: implemented by the solint parameter in fringefit.

\item It must be possible to request a single fringe solution on each
  and every scan.

  Note: implemented in fringefit though generic CASA data selection.

\item It must be possible to request fringe solutions spanning more
  than one scan.

  Note: implemented in fringefit though generic CASA data selection.

\item It should be possible to combine data by IF and/or polarization
  prior to determination of the delay paramters.

  Note: implemented in fringefit through generic CASA calibration
  infrastructure.  Not sure if/how combining by polarization works.

\item It shall be possible for the user to specify delay and rate
  windows in which to search.

  Note: Implemented in fringefit, but parameters are not per-antenna.

\item It must be possible for the user to specify the minimum SNR of
  acceptable solutions with a sensible default value.

  Note: Implemented in fringefit.

\end{enumerate}

\subsection{Polarization}

\subsubsection{Polarization Calibration}

\begin{enumerate}[subsubseclist]

\item It must be possible to correct observed visibility phases for
  field rotation angle (parallactic angle and elevation angle)
  effects.

  Note: Implemented through parang parameter in various calibration
  tasks.  Support for Nasmyth mounts has been implemented.  X-Y mount
  still missing.

\end{enumerate}

\subsubsection{Polarization Plotting}

Note: Implemented in plotms.

\subsubsection{Polarization Basis Conversion}

Note: Implemented in Ivan Mart-Vidal's polconvert package.


\subsection{Model accountability and manipulation}

\subsubsection{Delay model adjustments}

\begin{enumerate}[subsubseclist]

\item It shall be possible to correct for antenna position errors.

  Note: Implemented in the \texttt{gencal} task.

  \begin{enumerate}[subsubsecsublist]

  \item New antenna position shall be specified as new ITRF frame
    coordinates for one or more antennas.

  \item Antenna position corrections up to 10 meters should be supported.

  \item The antenna position as reported in the antenna table shall
    be updated.

    Question: Are the positions actually updated?

  \end{enumerate}

\item Manual delay and rate adjustment shall be supported

  Note: This is certainly possible by creating calibration tables
  using the Python interfaces.  Support in the \texttt{gencal} task
  may be desirable.

\end{enumerate}

\subsection{Data export}

\subsubsection{Data}

\begin{enumerate}[subsubseclist]

\item It shall be possible to export data in AIPS FITS (UVFITS) format.

  Note: Implemented in exportuvfits.
  
\item It should be possible to export data in a well-defined text format.

  Note: CASA tables can be written out into text format using a python
  interface.
  
\end{enumerate}

\subsubsection{Calibration}

\begin{enumerate}[subsubseclist]
  
\item It should be possible to export calibration information in a
  well-defined text format.

\end{enumerate}


\section{Dropped Requirements}

\subsection{Data Import}

\subsubsection{Data selection}

\begin{enumerate}[subseclist]

\item Selection during import shall be supported.  Where appropraite
  data selection should apply to calibration tables as well as
  visibility data.

  Rationale: Data selection during import is not something that CASA
  supports in general (with the exception of selecting scans in
  importasdm).  CASA has powerfull tools for data selection at various
  other stages.  For example, the \texttt{mstransform} task can be
  used to create a Measurement Set based on a subset of the data.

\item Precision and data dimensions

  Rationale: CASA meets all these requirements, although subarray
  support is largely implicit.

\end{enumerate}

\subsection{Amplitude calibration}

\subsubsection{Primary beam correction}

\begin{enumerate}[subsubseclist]

\item A correction for the antenna primary beam based on the vector
  offset between the antenna pointing center and the correlator phase
  center shall be possible.

\item It should be possible to correct for the primary beam without
  detailed user input, either through built in models or through
  automated download.

\item It should be possible for a user to supply a detailed primary
  beam model tabulated in two dimensions.

\item It should be possible for a user to supply a simple radial
  polynomial model.

\item Primary beam corrections for phased arrays shall be possible.

  \begin{enumerate}[subsubsecsublist]

  \item Input parameters describing a general elliptical Bessel
    or Gaussian zenith beam shall be supported.

  \item This zenith beam shall be appropriately stretched as a
    function of time to account for fore-shortening of the array as seen by the source.

  \item The beam parameters should be scaled in angle with observing
    wavelength during application.

  \end{enumerate}

\end{enumerate}

Note: This is probably best implemented in a pipeline such as rPICARD
based on the building blocks already provided by CASA.

\subsubsection{Autocorrelation template fitting}

Was used in the past to measure antenna gains by comparing
autocorrelation spectra on masters.  Not clear if this is still used.

\subsection{Fringe Fitting}

\subsubsection{Modes of operation}

\begin{enumerate}[subsubseclist]

\item \[low priority\] It shall be possible to fit delay as a spline
  function over a period of time possibly greater than the coherence
  time.

  \begin{enumerate}[subsubsecsublist]

  \item The user shall have control over the spline degrees of freedom.

  \item Continuity of the spline across scan boundaries must be selectable.

  \item It should be possible to construct a spline solution based on
    tabulated solutions.

  \end{enumerate}

\end{enumerate}

Note: Short coherence time algorithms have been implemented in rPICARD.

\subsubsection{Visualization}

Considered out of scope of the current effort.  Probably should be
done as a seprate tool, potentially as an extension to jiveplot.

\subsection{Short coherence time algorithms}

Note: Short coherence time algorithms have been implemented in rPICARD.

\subsection{Data export}

\subsubsection{Data}

\begin{enumerate}[subsubseclist]
  \item It must be possible to export data, including visibilities and
    closure quantities.

  \item It shall be possible to export data in OIFITS format.

  \item It shall be possible to export data in Mark4 format.
    
\end{enumerate}

\end{document}
